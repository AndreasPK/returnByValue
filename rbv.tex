\documentclass[acmsmall,review,anonymous]{acmart}\settopmatter{printfolios=true,printccs=false,printacmref=false}

\acmJournal{PACMPL}
\acmVolume{1}
\acmNumber{CONF} % CONF = POPL or ICFP or OOPSLA
\acmArticle{1}
\acmYear{2018}
\acmMonth{1}
\acmDOI{} % \acmDOI{10.1145/nnnnnnn.nnnnnnn}
\startPage{1}

\setcopyright{rightsretained}

%% Bibliography style
\bibliographystyle{ACM-Reference-Format}
%% Citation style
%% Note: author/year citations are required for papers published as an
%% issue of PACMPL.
\citestyle{acmauthoryear}   %% For author/year citations


%% Some recommended packages.
\usepackage{booktabs}   %% For formal tables:
                        %% http://ctan.org/pkg/booktabs
\usepackage{subcaption} %% For complex figures with subfigures/subcaptions
                        %% http://ctan.org/pkg/subcaption


\begin{document}

%% Title information
\title[Return By Value]{Call by Name, Return by Value}         %% [Short Title] is optional;
                                        %% when present, will be used in
                                        %% header instead of Full Title.
\subtitle{Knowing when the work is already done}                     %% \subtitle is optional


\author{Andreas Klebinger}
\affiliation{
  \position{Position1}
  \department{Department1}              %% \department is recommended
  \institution{Institution1}            %% \institution is required
  \streetaddress{Street1 Address1}
  \city{City1}
  \state{State1}
  \postcode{Post-Code1}
  \country{Country1}                    %% \country is recommended
}
\email{first1.last1@inst1.edu}          %% \email is recommended

\author{Jos\'{e} Manuel Calder\'{o}n Trilla}
\affiliation{
  \institution{Galois, Inc.}            %% \institution is required
}
\email{jmct@jmct.cc}          %% \email is recommended


\begin{abstract}
An abstract is important, it helps the reader know if they paper is worth their investment of time.
It is important to `sell' the paper, but be careful not to \emph{oversell}.

\end{abstract}

\keywords{optimization, lazy functional languages, haskell, static analysis}

\maketitle

\section{Introduction}
\label{sec:intro}
The introduction sets the stage of the paper.
We want to provide a high-level view of the work, the way you would explain it to someone in person over a few minutes.
This way, if they only read the introduction, they will have a good idea about what the paper is about.


\section{Avoiding Evaluation Checks}
\label{sec:optimization}
This section provides a deeper look into the problem of recurring checks on evaluation.
In particular, we would like to make it especially clear why it is so expensive.
Examples will be the name of the game.
Each example should show how GHC is `obviously' missing out by not having this analysis and optimization already.


% I think a section should go here that gives a quick review of STG
% and what constructs are necessary to know
\section{STG, In Brief}
\label{sec:stg}
This section contains what is on the tin.
We cannot assume that every reader is up-to-date on how STG works.
Examples will be useful as well as an emphasis on the operational nature of STG.


\section{Status Analysis}
\label{sec:analysis}
Here we present an idealized form of the analysis.
This will require discussion of the chosen domain and how to traverse the AST.
At the end of this section someone should be able to analyze a function by-hand, even if they would not know how to implement the analysis in GHC.


\section{Transformation}
\label{sec:transform}
Now that we have presented the analysis, how does the back-end utilize this new information?
This section should answer that concretely.
Use examples!


\section{Evaluation}
\label{sec:evaluation}
\input{evaluation}

\section{Discussion}
\label{sec:discussion}
\input{discussion}

\section{Related Work}
\label{sec:related-work}
\input{related-work}

\section{Conclusions and Future Work}
\label{sec:conclusions}
\input{conclusion}

\end{document}
